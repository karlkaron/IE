\chapterimage{ddd.jpg} % Chapter heading image

\chapter{Istoria Medievală este}
\epigraph{Corrige praetertum, praesens rege, cerne futurum - Analizeaza trecutul, conducete de prezent, prevede viitorul }{Lucius Annaeus Seneca minor}
\section{Ce este istoria ?}\index{Ce este istoria ?}

Interesul faţă de trecut este specific rasei umane. Acest interes este greu de explicat doar prin o curiozitate umană. Faptul că omul însuşi  este o fiinţă istorică.  El se dezvoltă şi să schimbă în timp, fiind produsul a acestei dezvoltări.

Sensul originar al cuvântului "istorie" vine din vechea  greacă  ce insemna "anchetă", "recunoaştere", "înfiinţarea". În antichitate cuvântului "istoria" era folosit ca determinarea cunoştinţelor obţinute prin cercetări în general, adică nu numai despre evenimentele din trecut cum suntem deprişi să folosim acest cuvînt astăzi. Astfel, Aristotel a folosit acest cuvînt în "Istoria animalelor". De asemenea, îl găsim în imnurile lui Homer, scrierile lui Heraclit şi în textul jurîmântului statului atenian.



În mitologia greacă, patronatul asupra istoriei era asigurat de una din muze Clio- o femeie tânără, cu o faţă înspirată, cu un papirus sau pergament în mînă. Numele  muzei Clio - este o derivată  a cuvîntului  grecesc "laudă".



\section{Istoria ca ştiinţă}\index{Istoria ca ştiinţă}
\epigraph{Historia est Magistra Vitae}{Marcus Tullius Cicero}
After I arrived and had my first meeting with Pauline, she explained me a general idea of what she wanted and shared me some more papers (about multi-wavelenght studies), I read the information and came up with the objective.

\begin{itemize}
\item Find out a method to transform data from a high dimensional dataset (FITS cube or any other data arrangement) to a low dimensional understandable information (graphs, clusters).
\end{itemize}

This means that from multiple images with different wavelengths of the same target apply an algorithm to find the hidden patterns that lie hidden between them.

\section{Ştiinţele istorice}\index{Ştiinţele istorice}
Ok, here is where I explain from where this is going to start, at that time I just had a microcontrollers and engineering design course my mind was set completelly to find appplicable theories and create uselful things with them, which is the complete opposite of how astronomy works. First, there's no way to test an experiment with galaxies and most of the information is fuzzy and subjective (not all). The process of having an, let's say \emph{astronomy idea} is a result of applying all your physics knowledge and consider the \textbf{cosmological principle},
\begin{quote}
The (testable) assumption that the same physical laws that apply here and now also apply everywhere and at all times, and that there are no special locations or directions in the universe.
\end{quote}

That's how science is made, thinking and testing and thinking again, creating your own scientific method, comming up with hypothesis, learning what might work and what not, using your insticts. 

Well, before comming here I didn't think like that, it was just all about being super productive and thinking about doing robots and all kinds of devices with sensors. I had some experience programming in C/C++, no computer science backgound and I had never had an astronomy course.

This report was written in order to help someone to continue researching about data mining techniques applied in Astronomy, I explain how did I come up with the clustering techniques, my hypothesis, some tests and other ideas I have had, I hope this can help anyone and the research is continued. Anything you may need/questions do not hesitate to contact me, my e-mail address is: \emph{mrs.petzl@gmail.com}, also s part of my own documentation I created a GitHub page where you can download all the codes I programmed and find more information. The link to this page is: \url{https://github.com/LaurethTeX/Clustering}, from the \textsc{readme} file you can acces to all the pages, take your time to surf.
%------------------------------------------------

\subsection{References}\index{References}

Since I found so much good information about pretty much everything I wanted to know about, I will just create a remark and let you know where you can find more specific information about, just like below.

\begin{remark}
For more information about the cosmological principle, review Chapter 1: Why Learn Astronomy?, page 10, from \textbf{21st Century Astronomy}, \textit{Hester | Smith | Blumenthal | Kay | Voss}, Third Edition, 2010.
\end{remark}
\begin{expli}
For more information about the cosmological principle, review Chapter 1: Why Learn Astronomy?, page 10, from \textbf{21st Century Astronomy}, \textit{Hester | Smith | Blumenthal | Kay | Voss}, Third Edition, 2010.
\end{expli}


%This statement requires citation \cite{book_key}; this one is more specific \cite[122]{article_key}.
